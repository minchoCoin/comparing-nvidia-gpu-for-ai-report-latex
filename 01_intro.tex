\section{서론}
그림 인공지능(AI Image Generator)과 거대 언어 모델(Large Language Model)은 인공지능
시대를 연 인공지능의 대표분야들이다. 그림 인공지능은 생성적 적대 신경망(Generative 
Adversarial Network)를 이용하여 이미지를 생성하는 인공지능 기술이다. 그리고 거대 언어
모델은 수많은 파라미터를 보유한 인공 신경망으로 구성되는 언어 모델\cite{wiki:llm} (입력된
자연어를 기반으로 가장 적절한 글자를 출력하는 모델)이다.

그러나 이러한 모델을 돌리기 위해서는 많은 GPU 연산과 GPU 메모리가 필요하다. 그래서
많은 그림 인공지능이나 거대 언어 모델들이 온라인으로 서비스하고 있다. 그림
인공지능에서는 DALL-E\cite{ramesh2021zeroshot}, DALL-E2\cite{ramesh2022hierarchical}, Novel ai\cite{novelAIDocu} 등이 있다. 거대 언어 모델에는 Open 
AI 사의 ChatGPT 가 있다.

그러나 많은 서비스들이 유료이고, 입력 데이터를 모델 개선에 활용하는 경우, 민감한
데이터 유출의 위험도 존재한다. 실제로 ChatGPT 의 입력 데이터는 저장되며 모델 개선에
활용될 수 있다고 OpenAI 사의 개인정보정책에 명시 되어있다\cite{kisa2023chatgpt}. 따라서 회사 등에서
이러한 생성 인공지능을 활용할 때는 로컬에서 돌릴 필요가 있을 것이다. 이때 GPU 가
필요한데, 어떤 GPU 가 성능이 좋고 가격 대비 성능이 좋은 지, 그래픽카드별 성능차이를
살펴보고자 한다.

